% Created 2014-05-15 Thu 09:23
\documentclass[11pt]{article}
\usepackage[utf8]{inputenc}
\usepackage[T1]{fontenc}
\usepackage{fixltx2e}
\usepackage{graphicx}
\usepackage{longtable}
\usepackage{float}
\usepackage{wrapfig}
\usepackage{soul}
\usepackage{textcomp}
\usepackage{marvosym}
\usepackage{wasysym}
\usepackage{latexsym}
\usepackage{amssymb}
\usepackage{hyperref}
\tolerance=1000
\providecommand{\alert}[1]{\textbf{#1}}

\title{OpenFOAM C++ style guide}
%\author{OpenFOAM Foundation}
\date{Aug 2011}
\hypersetup{
  pdfkeywords={},
  pdfsubject={},
  pdfcreator={Emacs Org-mode version 7.8.03}}

\begin{document}

\maketitle

\setcounter{tocdepth}{3}
\tableofcontents
\vspace*{1cm}

\section{OpenFOAM C++ style guide}
\label{sec-1}
\subsection{General}
\label{sec-1-1}

\begin{itemize}
\item 80 character lines max
\item The normal indentation is 4 spaces per logical level.
\item Use spaces for indentation, not tab characters.
\item Avoid trailing whitespace.
\item The body of control statements (eg, \texttt{if}, \texttt{else}, \texttt{while}, etc). is
      always delineated with brace brackets. A possible exception can be
      made in conjunction with \texttt{break} or \texttt{continue} as part of a control
      structure.
\item The body of \texttt{case} statements is usually delineated with brace brackets.
\item A fall-through \texttt{case} should be commented as such.
\item stream output
\begin{itemize}
\item \texttt{<<} is always four characters after the start of the stream,
        so that the \texttt{<<} symbols align, i.e.

\begin{verbatim}
Info<< ...
os  << ...
\end{verbatim}
        so

\begin{verbatim}
WarningIn("className::functionName()")
    << "Warning message"
\end{verbatim}
        \textbf{not}

\begin{verbatim}
WarningIn("className::functionName()")
<< "Warning message"
\end{verbatim}
\end{itemize}
\item no unnecessary class section headers, i.e. remove
\end{itemize}

\begin{verbatim}
// * * * * * * * * * * * * * Private Member Functions  * * * * * * * * * * * //

    // Check

    // Edit

    // Write
\end{verbatim}
      if they contain nothing, even if planned for `future use'

\begin{itemize}
\item class titles are centred
\end{itemize}

\begin{verbatim}
/*---------------------------------------------------------------------------*\
                        Class exampleClass Declaration
\*---------------------------------------------------------------------------*/
\end{verbatim}

      \textbf{not}


\begin{verbatim}
/*---------------------------------------------------------------------------*\
                Class exampleClass Declaration
\*---------------------------------------------------------------------------*/
\end{verbatim}
\subsection{The \emph{.H} Files}
\label{sec-1-2}

\begin{itemize}
\item header file spacing
\begin{itemize}
\item Leave two empty lines between sections
        (as per functions in the \emph{.C} file etc)
\end{itemize}
\item use \texttt{//- Comment} comments in header file to add descriptions to class
      data and functions do be included in the Doxygen documentation:
\begin{itemize}
\item text on the line starting with \texttt{//-} becomes the Doxygen brief
        description;
\item text on subsequent lines becomes the Doxygen detailed description \emph{e.g.}

\begin{verbatim}
//- A function which returns a thing
//  This is a detailed description of the function
//  which processes stuff and returns other stuff
//  depending on things.
thing function(stuff1, stuff2);
\end{verbatim}
\item list entries start with \texttt{-} or \texttt{-\#} for numbered lists but cannot start
        on the line immediately below the brief description so

\begin{verbatim}
//- Compare triFaces
//  Returns:
//  -  0: different
//  - +1: identical
//  - -1: same face, but different orientation
static inline int compare(const triFace&, const triFace&);
\end{verbatim}
        or

\begin{verbatim}
//- Compare triFaces returning 0, +1 or -1
//
//  -  0: different
//  - +1: identical
//  - -1: same face, but different orientation
static inline int compare(const triFace&, const triFace&);
\end{verbatim}
        \textbf{not}

\begin{verbatim}
//- Compare triFaces returning 0, +1 or -1
//  -  0: different
//  - +1: identical
//  - -1: same face, but different orientation
static inline int compare(const triFace&, const triFace&);
\end{verbatim}
\item list can be nested for example

\begin{verbatim}
//- Search for \em name
//  in the following hierarchy:
//  -# personal settings:
//    - ~/.OpenFOAM/\<VERSION\>/
//      <em>for version-specific files</em>
//    - ~/.OpenFOAM/
//      <em>for version-independent files</em>
//  -# site-wide settings:
//    - $WM_PROJECT_INST_DIR/site/\<VERSION\>
//      <em>for version-specific files</em>
//    - $WM_PROJECT_INST_DIR/site/
//      <em>for version-independent files</em>
//  -# shipped settings:
//    - $WM_PROJECT_DIR/etc/
//
//  \return the full path name or fileName() if the name cannot be found
//  Optionally abort if the file cannot be found
fileName findEtcFile(const fileName&, bool mandatory=false);
\end{verbatim}
\item for more details see the Doxygen documentation.
\end{itemize}
\item destructor
\begin{itemize}
\item If adding a comment to the destructor -
        use \texttt{//-} and code as a normal function:

\begin{verbatim}
//- Destructor
~className();
\end{verbatim}
\end{itemize}
\item inline functions
\begin{itemize}
\item Use inline functions where appropriate in a separate \emph{classNameI.H}
        file.  Avoid cluttering the header file with function bodies.
\end{itemize}
\end{itemize}
\subsection{The \emph{.C} Files}
\label{sec-1-3}

\begin{itemize}
\item Do not open/close namespaces in a \emph{.C} file
\begin{itemize}
\item Fully scope the function name, i.e.

\begin{verbatim}
Foam::returnType Foam::className::functionName()
\end{verbatim}
        \textbf{not}

\begin{verbatim}
namespace Foam
{
    ...
    returnType className::functionName()
    ...
}
\end{verbatim}
        EXCEPTION

        When there are multiple levels of namespace, they may be used in the
        \emph{.C} file, i.e.

\begin{verbatim}
namespace Foam
{
namespace compressible
{
namespace RASModels
{
    ...
} // End namespace RASModels
} // End namespace compressible
} // End namespace Foam
\end{verbatim}
\end{itemize}
\item Use two empty lines between functions
\end{itemize}
\subsection{Coding Practice}
\label{sec-1-4}

\begin{itemize}
\item passing data as arguments or return values.
\begin{itemize}
\item Pass bool, label and scalar as copy, anything larger by reference.
\end{itemize}
\item const
\begin{itemize}
\item Use everywhere it is applicable.
\end{itemize}
\item variable initialisation using

\begin{verbatim}
const className& variableName = otherClass.data();
\end{verbatim}
      \textbf{not}

\begin{verbatim}
const className& variableName(otherClass.data());
\end{verbatim}
\item virtual functions
\begin{itemize}
\item If a class is virtual, make all derived classes virtual.
\end{itemize}
\end{itemize}
\subsection{Conditional Statements}
\label{sec-1-5}


\begin{verbatim}
if (condition)
{
    code;
}
\end{verbatim}
    OR

\begin{verbatim}
if
(
   long condition
)
{
    code;
}
\end{verbatim}
    \textbf{not} (no space between \texttt{if} and \texttt{(} used)

\begin{verbatim}
if(condition)
{
    code;
}
\end{verbatim}
\subsection{\texttt{for} and \texttt{while} Loops}
\label{sec-1-6}


\begin{verbatim}
for (i = 0; i < maxI; i++)
{
    code;
}
\end{verbatim}
    OR

\begin{verbatim}
for
(
    i = 0;
    i < maxI;
    i++
)
{
    code;
}
\end{verbatim}
    \textbf{not} this (no space between \texttt{for} and \texttt{(} used)

\begin{verbatim}
for(i = 0; i < maxI; i++)
{
    code;
}
\end{verbatim}
    Note that when indexing through iterators, it is often slightly more
    efficient to use the pre-increment form. Eg, \texttt{++iter} instead of \texttt{iter++}
\subsection{\texttt{forAll}, \texttt{forAllIter}, \texttt{forAllConstIter}, etc. loops}
\label{sec-1-7}

    like \texttt{for} loops, but

\begin{verbatim}
forAll(
\end{verbatim}
    \textbf{not}

\begin{verbatim}
forAll (
\end{verbatim}
    Using the \texttt{forAllIter} and \texttt{forAllConstIter} macros is generally
    advantageous - less typing, easier to find later.  However, since
    they are macros, they will fail if the iterated object contains
    any commas.

    The following will FAIL!:


\begin{verbatim}
forAllIter(HashTable<labelPair, edge, Hash<edge> >, foo, iter)
\end{verbatim}
    These convenience macros are also generally avoided in other
    container classes and OpenFOAM primitive classes.
\subsection{Splitting Over Multiple Lines}
\label{sec-1-8}
\subsubsection{Splitting return type and function name}
\label{sec-1-8-1}

\begin{itemize}
\item split initially after the function return type and left align
\item do not put \texttt{const} onto its own line - use a split to keep it with
        the function name and arguments.

\begin{verbatim}
const Foam::longReturnTypeName&
Foam::longClassName::longFunctionName const
\end{verbatim}
        \textbf{not}

\begin{verbatim}
const Foam::longReturnTypeName&
    Foam::longClassName::longFunctionName const
\end{verbatim}
        \textbf{nor}

\begin{verbatim}
const Foam::longReturnTypeName& Foam::longClassName::longFunctionName
const
\end{verbatim}
        \textbf{nor}

\begin{verbatim}
const Foam::longReturnTypeName& Foam::longClassName::
longFunctionName const
\end{verbatim}
\item if it needs to be split again, split at the function name (leaving
        behind the preceding scoping =::=s), and again, left align, i.e.

\begin{verbatim}
const Foam::longReturnTypeName&
Foam::veryveryveryverylongClassName::
veryveryveryverylongFunctionName const
\end{verbatim}
\end{itemize}
\subsubsection{Splitting long lines at an ``=''}
\label{sec-1-8-2}

     Indent after split

\begin{verbatim}
variableName =
    longClassName.longFunctionName(longArgument);
\end{verbatim}
     OR (where necessary)

\begin{verbatim}
variableName =
    longClassName.longFunctionName
    (
        longArgument1,
        longArgument2
    );
\end{verbatim}
     \textbf{not}

\begin{verbatim}
variableName =
longClassName.longFunctionName(longArgument);
\end{verbatim}
     \textbf{nor}

\begin{verbatim}
variableName = longClassName.longFunctionName
(
    longArgument1,
    longArgument2
);
\end{verbatim}
\subsection{Maths and Logic}
\label{sec-1-9}

\begin{itemize}
\item operator spacing

\begin{verbatim}
a + b, a - b
a*b, a/b
a & b, a ^ b
a = b, a != b
a < b, a > b, a >= b, a <= b
a || b, a && b
\end{verbatim}
\item splitting formulae over several lines

      Split and indent as per ``splitting long lines at an =''
      with the operator on the lower line.  Align operator so that first
      variable, function or bracket on the next line is 4 spaces indented i.e.

\begin{verbatim}
variableName =
    a*(a + b)
   *exp(c/d)
   *(k + t);
\end{verbatim}
      This is sometimes more legible when surrounded by extra parentheses:


\begin{verbatim}
variableName =
(
    a*(a + b)
   *exp(c/d)
   *(k + t)
);
\end{verbatim}
\item splitting logical tests over several lines

      outdent the operator so that the next variable to test is aligned with
      the four space indentation, i.e.

\begin{verbatim}
if
(
    a == true
 && b == c
)
\end{verbatim}
\end{itemize}
\subsection{General}
\label{sec-1-11}

\begin{itemize}
\item For readability in the comment blocks, certain tags are used that are
      translated by pre-filtering the file before sending it to Doxygen.
\item The tags start in column 1, the contents follow on the next lines and
      indented by 4 spaces. The filter removes the leading 4 spaces from the
      following lines until the next tag that starts in column 1.
\item The `Class' and `Description' tags are the most important ones.
\item The first paragraph following the `Description' will be used for the
      brief description, the remaining paragraphs become the detailed
      description.

      For example,

\begin{verbatim}
Class
    Foam::myClass

Description
    A class for specifying the documentation style.

    The class is implemented as a set of recommendations that may
    sometimes be useful.
\end{verbatim}
\item The class name must be qualified by its namespace, otherwise Doxygen
      will think you are documenting some other class.
\item If you don't have anything to say about the class (at the moment), use
      the namespace-qualified class name for the description. This aids with
      finding these under-documented classes later.

\begin{verbatim}
Class
    Foam::myUnderDocumentedClass

Description
    Foam::myUnderDocumentedClass
\end{verbatim}
\item Use `Class' and `Namespace' tags in the header files.
      The Description block then applies to documenting the class.
\item Use `InClass' and `InNamespace' in the source files.
      The Description block then applies to documenting the file itself.

\begin{verbatim}
InClass
    Foam::myClass

Description
    Implements the read and writing of files.
\end{verbatim}
\end{itemize}
\subsection{Doxygen Special Commands}
\label{sec-1-12}

    Doxygen has a large number of special commands with a =\= prefix.

    Since the filtering removes the leading spaces within the blocks, the
    Doxygen commmands can be inserted within the block without problems.

\begin{verbatim}
InClass
    Foam::myClass

Description
    Implements the read and writing of files.

    An example input file:
    \verbatim
        patchName
        {
            type        myPatchType;
            refValue    100;
            value       uniform 1;
        }
    \endverbatim

    Within the implementation, a loop over all patches is done:
    \code
        forAll(patches, patchI)
        {
            ...  // some operation
        }
    \endcode
\end{verbatim}
\subsection{HTML Special Commands}
\label{sec-1-13}

    Since Doxygen also handles HTML tags to a certain extent, the angle
    brackets need quoting in the documentation blocks. Non-HTML tags cause
    Doxygen to complain, but seem to work anyhow.

    eg,
\begin{itemize}
\item The template with type \texttt{<HR>} is a bad example.
\item The template with type \texttt{\textbackslash{}<HR\textbackslash{}>} is a better example.
\item The template with type \texttt{<Type>} causes Doxygen to complain about an
      unknown html type, but it seems to work okay anyhow.
\end{itemize}
\subsection{Documenting Namespaces}
\label{sec-1-14}

\begin{itemize}
\item If namespaces are explictly declared with the \texttt{Namespace()} macro,
      they should be documented there.
\item If the namespaces is used to hold sub-models, the namespace can be
      documented in the same file as the class with the model selector.
      eg,

\begin{verbatim}
documented namespace 'Foam::functionEntries' within the
class 'Foam::functionEntry'
\end{verbatim}
\item If nothing else helps, find some sensible header.
      eg,

\begin{verbatim}
namespace 'Foam' is documented in the foamVersion.H file
\end{verbatim}
\end{itemize}
\subsection{Documenting typedefs and classes defined via macros}
\label{sec-1-15}

    \ldots{} not yet properly resolved
\subsection{Documenting Applications}
\label{sec-1-16}

    Any number of classes might be defined by a particular application, but
    these classes will not, however, be available to other parts of
    OpenFOAM. At the moment, the sole purpuse for running Doxygen on the
    applications is to extract program usage information for the `-doc'
    option.

    The documentation for a particular application is normally contained
    within the first comment block in a \emph{.C} source file. The solution is this
    to invoke a special filter for the ``/applications/\{solver,utilities\}/''
    directories that only allows the initial comment block for the \emph{.C} files
    through.

    The layout of the application documentation has not yet been finalized,
    but foamToVTK shows an initial attempt.
\subsection{Orthography}
\label{sec-1-17}

    Given the origins of OpenFOAM, the British spellings (eg, neighbour and not
    neighbor) are generally favoured.

    Both `-ize' and the `-ise' variant are found in the code comments. If
    used as a variable or class method name, it is probably better to use
    `-ize', which is considered the main form by the Oxford University
    Press. Eg,

\begin{verbatim}
myClass.initialize()
\end{verbatim}

    The word ``its'' (possesive) vs. ``it's'' (colloquial for ``it is'' or ``it has'')
    seems to confuse non-native (and some native) English speakers.
    It is better to donate the extra keystrokes and write ``it is'' or ``it has''.
    Any remaining ``it's'' are likely an incorrect spelling of ``its''.

\end{document}
